%
% File acl2017.tex
%
%% Based on the style files for ACL-2015, with some improvements
%%  taken from the NAACL-2016 style
%% Based on the style files for ACL-2014, which were, in turn,
%% based on ACL-2013, ACL-2012, ACL-2011, ACL-2010, ACL-IJCNLP-2009,
%% EACL-2009, IJCNLP-2008...
%% Based on the style files for EACL 2006 by 
%%e.agirre@ehu.es or Sergi.Balari@uab.es
%% and that of ACL 08 by Joakim Nivre and Noah Smith

\documentclass[11pt,a4paper]{article}
\usepackage[hyperref]{acl2017}
\usepackage{times}
\usepackage{latexsym}
\usepackage{graphicx}
\usepackage{booktabs}
\usepackage{amsmath}
\usepackage{todonotes}
\usepackage{multirow}
\usepackage{enumitem}
\usepackage{amsfonts}
\usepackage{url}

\makeatletter
\newcommand{\@emptybiblabel}[1]{}
\makeatother
\DeclareMathOperator*{\argmax}{arg\,max}
\setlength\titlebox{3.5cm}    % Expanding the titlebox

\newcommand*{\checktikz}[1][]{\tikz[x=1em, y=1em]\fill[#1] (0,.35) -- (.25,0) -- (1,.7) -- (.25,.15) -- cycle;}
\newcommand*{\crosstikz}[1][]{\tikz[x=1em, y=1em]\fill[#1] (0,0) -- (1,1) -- (0.5,0.5) -- (0.1,0.1) -- cycle;}

\newcommand\BibTeX{B{\sc ib}\TeX}
\newcommand{\Correct}{\checktikz[draw=black]}
\newcommand{\ValidMiss}{\checktikz[draw=gray,fill=white]}
\newcommand{\Valid}{\checktikz[draw=gray,fill=white]}
\newcommand{\Missed}{\checktikz[draw=black]} %\textsf{X}~}
\newcommand{\Wrong}{} %\textsf{X}~}

\newcommand{\U}{\mathbb{U}}

\title{Are you asking the right questions? \\ Automatically Generating Clarification Questions}

\date{}

\begin{document}
\maketitle
\begin{abstract}
Inquiry is fundamental to communication, and machines cannot effectively collaborate with humans unless they can ask questions. We build neural network models inspired by the idea of the expected value of perfect information: a good question is one whose expected answer is going to be most useful. We study the problem of question generation using data from StackExchange, a plentiful online resource in which people routinely ask clarifying questions to posts so that they can better offer assistance to the original poster. We evaluate our model on three domains of StackExchange: askubuntu, unix and superuser and demonstrate significant improvements by modeling the value of information.
\end{abstract}

\section{Introduction}\label{introduction}

A main goal of asking questions is to fill information gaps, typically through clarification questions, which naturally occur in conversations. 
A good question is one whose \emph{likely answer} is going to be most useful.
Consider the exchange in Figure~\ref{askubuntu_post}, in which an initial poster (who we'll call ``Terry'') asks for help configuring environment variables.
This question is underspecified and a responder (``Parker'') asks a clarifying question ``\textsf{\small (a) What version of Ubuntu do you have?}''
Parker could alternatively have asked one of:

\textsf{\small(b) Is the moon waxing or waning?}

\textsf{\small(c) Are you running Ubuntu 14.10 kernel 4.4.0-59-generic on an x86\_64 architecture?}

\noindent
Parker should not ask (b) because it's not useful; they should not ask (c) because it's too specific and an answer of ``No'' gives little help.
Parker's question (a) is optimal: it is both likely to be useful, and is plausibly answerable by Terry.
Our goal in this work is to automate Parker.
Specifically, after Terry writes their initial post, we aim to generate a clarification question so that Terry can immediately amend their post in hopes of getting faster and better replies.
\begin{figure}[!t]
	\centering
	\setlength\fboxsep{1pt}
	\setlength\fboxrule{0.5pt}
	\fbox{\includegraphics[width=0.47\textwidth]{askubuntu_post}}
	\caption{A post on an online Q \& A forum ``askubuntu.com'' is updated to fill the missing information pointed out by the question comment}
	\label{askubuntu_post}
\end{figure}
%
Our work has two main contributions: 
\begin{enumerate}[noitemsep,nolistsep]
	%\item Identifying the problem of generating clarification questions as a problem worth of study, both in its own right and as part of the larger problem of building naturalistic conversational systems. 
	\item A novel neural-network model for addressing this task that integrates the notion of expected value of perfect information (\S\ref{model}). % , a classic formalization from AI
	\item A novel dataset, derived from StackExchange, that enables us to learn a model to ask clarifying questions by looking at the types of questions people ask (\S\ref{dataset_creation}). %\footnote{We use data from StackExchange; per license cc-by-sa 3.0, the data is ``intended to be shared and remixed'' (with attribution). We will release all of the data we extract.}
\end{enumerate}

\begin{figure*}[t]
	\centering
	\includegraphics[width=0.85\textwidth]{model}
	\caption{\small The behavior of our model during test time. Given a post $p$, Lucene retrieves 9 posts similar to post $p$ and consider the questions asked to those 9 posts, plus the original, as question candidates. The edits made to the posts in response to the questions as our answer candidates. For each question candidate $q_i$, we generate an answer representation $F(p,q_j)$ and calculate how close is the answer candidate $a_k$ to our answer representation $F(p,q_j)$. Our utility calculator calculates the utility of the post if it were updated with the answer $a_k$. Finally we return the question $q$ that maximizes the expected utility of the post $p$ (Equation~\ref{evpi_equation}).}
	\label{model}
\end{figure*}

The problem of question generation has received sparse attention from the natural language processing community. Most prior work focuses on generating reading comprehension questions:  given text, write questions that one might find on a standardized test \cite{vanderwende2008importance,heilman2011automatic,rus2011question,olney2012question}.  Comprehension questions, by definition, are answerable from the provided text. Clarification questions are not. 

\section{Model description}\label{model}

In order to choose what question to ask, we build a neural network model inspired by the theory of expected value of perfect information (EVPI). EVPI is a measurement of: if I were to acquire information X, how useful would that be to me? However, because we haven't acquired X yet, we have to take this quantity in expectation over all possible X, weighted by each X's likelihood. In the question generation setting, for any given question $q$ that we can ask, there is set $A$ of possible answers that could be given. For each possible answer $a \in A$, there is some probability of getting that answer, and some utility if that were the answer we got. The value of this question $q$ is the expected utility, over all possible answers. The theory of EVPI then states that we want to choose the question $q$ that maximizes:
\begin{equation}\label{evpi_equation}
\argmax_{q \in Q} \sum_{a \in A} \mathbb{P}[a | p,q] \U(p+a)
\end{equation} 

Figure~\ref{model} describes the behavior of our model during test time. 
Given a post $p$, we first generate a set of candidate questions $Q$ and a set of candidate answers $A$. Given a post $p$ and a question candidate $q_i$, we calculate how likely is this question to be answered using one of our answer candidates $a_k$ i.e. $\mathbb{P}[a_k | p,q_i]$. Given a post $p$ and an answer candidate $a_k$, we then calculate the utility of the updated post i.e. $\U(p + a_k)$ which measures how useful it would be if $p$ were augmented with answer $a_k$. Finally, using these pieces, we build a joint neural network that we can optimize end-to-end over our data.

\section{Dataset creation}\label{dataset_creation}

 We train our models jointly based on $(p,q,a)$ triples that we extract automatically from StackExchange.
Figure~\ref{askubuntu_post} depicts how we do this using StackExchange's edit history.  In the figure, the initial post fails to state what version of Ubuntu is being run. In response to Parker's question in the comments section, Terry, the author of the post, edits the post to answer Parker's clarification question. We extract the initial post as $p$, question posted in the comments section as $q$, and edit to the original post as answer $a$ to form our $(p,q,a)$ triples. We extract a total of 37K triples from three domains of StackExchange: askubuntu, unix and superuser.

\section{Results}

Our results show significant improvements from using the EVPI formalism over both standard feedforward network architectures and bag-of-ngrams baselines, even when our system builds on strong information retrieval scaffolding. We additionally analyze the difficulty of this task for non-expert humans, and give examples of system output. %on which our system does well, and several on which it does poorly.

We have described a new dataset for question generation, and attacked this problem from the perspective of question selection.
Our model integrates state-of-the-art neural network structure with the classic notion of expected value of perfect information, which effectively models a pragmatic choice on the part of the questioner: how do I \emph{imagine} the other party would answer if I were to ask this question. to make question generation more general, systems need to be able to generalize, for instance by constructing templates of the from ``What version of \_\_\_ are you running?'' into which the system would need to fill a variable. Finally, asking question is a natural component of dialog, and building a collaborative dialog system that can naturally converse with a user is a broad long term goal.

\bibliography{acl2017}
\bibliographystyle{acl_natbib}

\end{document}
